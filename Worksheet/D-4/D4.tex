\documentclass[11pt]{article}
\usepackage{hyperref}
\usepackage[normalem]{ulem}
\usepackage{color}
\hypersetup{colorlinks,breaklinks,
            linkcolor=blue,urlcolor=blue,
            anchorcolor=lue,citecolor=blue}

\usepackage{amsmath, verbatim}
\usepackage{amssymb}
\usepackage{fancyhdr}
\usepackage{pstricks}
\usepackage{pstricks-add}


\setlength{\oddsidemargin}{0.0in}
\setlength{\evensidemargin}{0.0in}
\setlength{\textheight}{8.4in}
\setlength{\textwidth}{6.5in}
\setlength{\voffset}{0.00in}
\setlength{\headsep}{26pt}
\setlength{\parindent}{0pt}
\setlength{\parskip}{6pt}

% header information
\pagestyle{fancyplain}
\lhead{MATH 150: Wednesday January 27}
\rhead{\textit{Spring Semester 2021}}



\chead{\textbf{Discussion 4:}}

\begin{document}

The goals of this discussion section are: 

\begin{enumerate}
\item Be able to study equilibrium of some dynamical systems
\item Identify a model of population dynamics
\end{enumerate}

Participation in discussion section counts as 5\% of the grade. Completion of the worksheets counts as 20\% of the grade. \textbf{Submit your worksheet work by February 17th at 2:59pm.}

The Lorenz equations form a simplified mathematical model for atmospheric convection. It is given by
\[ \begin{aligned}
& \frac{dX}{dt} = \sigma(Y -X) \\
&  \frac{dY}{dt}=  - XZ + \rho X - Y \\
&  \frac{dZ}{dt} = XY - \beta Z
\end{aligned}\]
with $\sigma$, $\rho$, $\beta$ some constants.
Those equations describe the rate of change of three quantities with respect to time: X is proportional to the rate of convection, Y to the horizontal temperature variation, and Z to the vertical temperature variation. The constants $\sigma$, $\rho$, $\beta$ are system parameters proportional to the Prandtl number, Rayleigh number, and certain physical dimensions of the atmospheric layer.
\begin{enumerate}
%where we will chose here $\sigma  = 10$, $\beta = 8/3 $ and $\rho$, proportional to $\Delta T$ will be the parameter to vary (aka test different values).
\item Make some observations about the system using this \href{https://alpha.iodide.io/notebooks/2275/?viewMode=report}{code}. In particular try the values given in Exercise 2.12 from \texttt{Math150\_Chapter2.pdf}
\item Discretize the Lorenz equations using Runge-Kutta methods (of order 2 or order 4). You may take inspiration from \href{https://matplotlib.org/3.2.1/gallery/mplot3d/lorenz_attractor.html}{this notebook} and adjust, or use techniques discussed in Remark 2.4 from  \texttt{Math150\_Chapter2.pdf}. Explain your choice.
\item Fixing $\sigma = 10$, $\beta  = \frac{8}{3}$, make observations for the following cases:
\begin{itemize}
\item test for $0 < \rho <1$
\item test for $0 < \rho <20$, $\rho=1$
\item test for $\rho = 28$
\end{itemize}
We suggest to create different code cells in your .ipynb for each test, and write comments below each simulation.
%\begin{enumerate}
\item Submit your work on Catcourses under the assignment \texttt{Worksheet 4} \textbf{as a .ipynb}. Do not forget to submit scans of the handwritten answers as well if they are not typed in the .ipynb.
\end{enumerate}

\end{document}
