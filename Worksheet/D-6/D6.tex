\documentclass[11pt]{article}


\usepackage{amsmath, verbatim}
\usepackage{amssymb}
\usepackage{fancyhdr}
\usepackage{pstricks}
\usepackage{pstricks-add}


\setlength{\oddsidemargin}{0.0in}
\setlength{\evensidemargin}{0.0in}
\setlength{\textheight}{8.4in}
\setlength{\textwidth}{6.5in}
\setlength{\voffset}{0.00in}
\setlength{\headsep}{26pt}
\setlength{\parindent}{0pt}
\setlength{\parskip}{6pt}

% header information
\pagestyle{fancyplain}
\lhead{MATH 150: Wednesday January 27}
\rhead{\textit{Spring Semester 2021}}



\chead{\textbf{Discussion 6: February 24}}
\begin{document}
The goals of this discussion section are: 

\begin{enumerate}
\item Implement random walkers in 1D.
\end{enumerate}

Participation in discussion section counts as 5\% of the grade. Completion of the worksheets counts as 20\% of the grade. \textbf{Submit your worksheet work by March 3rd at 2:59pm.}


The goal is to create a Python function that will simulate random walkers in two dimensions. The principle is the following: given $N$ random walkers, located in some square of size $a$ centered at $(x_0,y_0)$, we want to track their location in time. At each time-step $\Delta t$, the walkers can move left, right of length $\Delta x$, or up down of length $\Delta y$, according to some rules to set. \textbf{Work as a group to provide simulations. You may start with a 1D simulation then 2D simulation.}\\ 

Here are some questions and steps to help you create this script.
\begin{itemize}
\item What are the inputs of your function ? What are the outputs ?
\item Create a function that encapsulates the rules for one random walker in two dimensions (or one if you start with those cases). What is the easiest rules you could think of ?
\item How would you update the position of the walkers at each time-step ?
\item Make an animation of the walkers at each time step.
\end{itemize}

\begin{enumerate}
\item Once you defined your function, provide one example (of your choice), and provide your conjecture of the governing equation described by the random walkers. 
\item Submit your work on Catcourses under the assignment \texttt{Worksheet 6} \textbf{as a .ipynb}. 
\end{enumerate}

\end{document}
