\documentclass[11pt]{article}
\usepackage[normalem]{ulem}
\usepackage{color,hyperref}
\hypersetup{colorlinks,breaklinks,
            linkcolor=blue,urlcolor=blue,
            anchorcolor=lue,citecolor=blue}

\usepackage{amsmath, verbatim}
\usepackage{amssymb}
\usepackage{fancyhdr}
\usepackage{pstricks}
\usepackage{pstricks-add}


\setlength{\oddsidemargin}{0.0in}
\setlength{\evensidemargin}{0.0in}
\setlength{\textheight}{8.4in}
\setlength{\textwidth}{6.5in}
\setlength{\voffset}{0.00in}
\setlength{\headsep}{26pt}
\setlength{\parindent}{0pt}
\setlength{\parskip}{6pt}

% header information
\pagestyle{fancyplain}
\lhead{MATH 150: Wednesday January 27}
\rhead{\textit{Spring Semester 2021}}



\chead{\textbf{Discussion 1}}

\begin{document}
The goals of this first discussion section are: 

\begin{enumerate}
\item Install Anaconda on your computer and access Jupyter lab, or a Jupyter notebook.
\item Get familiar with the Jupyter notebook for numerical simulations 
%\item Acquire some sense of scales to use in a model
\item Create simple models (students should be creative !)
\end{enumerate}

Participation in discussion section counts as 5\% of the grade. Completion of the worksheets counts as 20\% of the grade. \textbf{Submit your worksheet work by January 27th at 2:59pm.}
\begin{enumerate}
\item Install Anaconda and launch a Jupyter notebook.
\begin{enumerate}
\item Download instructions can be found here \url{https://www.anaconda.com/products/individual}. We recommend to install Python 3.8 (update your Python package if you have installed it already).
\item Launch Anaconda Navigator and open a Jupyter notebook \url{https://docs.anaconda.com/anaconda/user-guide/getting-started/}. If you are comfortable with using a terminal, you may also launch it directly by executing the command \texttt{jupyter notebook}.
\item Open a Jupyter notebook, and play a little bit: let's code some basic Python commands ! Take a look at the user guide \url{https://jupyter-notebook.readthedocs.io/en/latest/notebook.html}. \\
We strongly recommend to take a look at the open source lessons from Software Carpentry \url{https://swcarpentry.github.io/python-novice-inflammation/}
\item Make a plot in your jupyter notebook. We strongly recommend to take a look at the open source lessons from Software Carpentry \url{http://swcarpentry.github.io/python-novice-gapminder/09-plotting/index.html}. You may for instance copy the first 2 cells given on this link.
\item Install nteract \url{https://nteract.io/}. This will allow you to view quickly .ipynb without launching Anaconda (very practical to view .ipynb files from email for example).
\end{enumerate}

%\item Suppose you have a company that makes cardboard boxes. Each box can contain $0.02 m^3$ of nuts. Each box has a square base, and double thickness at the top and bottom. Cardboard costs $\$ 0.30$ per square meter. Find the most economical size.
%\begin{enumerate}
%\item Start by a sketch and provide needed variables (unknowns). 
%\item Deduce quantities based on those variables.
%\item Find a formula for the cost then conclude. You may use the computer.
%\item Think about what you could do to improve the model (be creative).
%\end{enumerate}
%\item Create a simple mathematical model.
\item Watch the following video \url{https://www.youtube.com/watch?v=HW29067qVWk}. What is the video about ? What did you learn ? Write a small paragraph to summarize. \textit{A person who didn't watch the video should be able to understand your statement.}
\item Work on the exercises 1.1, 1.2, from the typed notes \texttt{Math150\_Chapter1.pdf}
\item Submit your work on Catcourses under the assignment \texttt{Worksheet 1} \textbf{as a single .ipynb} with all previous assignments in it. Use the markdown (or text) cell in the Jupyter Notebook to type your (non-code) answers.
\end{enumerate}
\end{document}
