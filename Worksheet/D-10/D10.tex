\documentclass[11pt]{article}
\usepackage{hyperref}
\usepackage[normalem]{ulem}
\usepackage{color}
\hypersetup{colorlinks,breaklinks,
            linkcolor=blue,urlcolor=blue,
            anchorcolor=lue,citecolor=blue}

\usepackage{amsmath, verbatim}
\usepackage{amssymb}
\usepackage{fancyhdr}
\usepackage{pstricks}
\usepackage{pstricks-add}


\setlength{\oddsidemargin}{0.0in}
\setlength{\evensidemargin}{0.0in}
\setlength{\textheight}{8.4in}
\setlength{\textwidth}{6.5in}
\setlength{\voffset}{0.00in}
\setlength{\headsep}{26pt}
\setlength{\parindent}{0pt}
\setlength{\parskip}{6pt}

% header information
\pagestyle{fancyplain}
\lhead{MATH 150: Wednesday January 27}
\rhead{\textit{Spring Semester 2021}}



\chead{\textbf{Discussion 10:}}

\begin{document}

The goals of this discussion section are: 

\begin{enumerate}
\item Work on your homework 7 group problem (early start compared to other times)
\item Get familiar with linear solvers
\end{enumerate}

Participation in discussion section counts as 5\% of the grade. Completion of the worksheets counts as 20\% of the grade. \textbf{Submit your worksheet work by April 7th at 2:59pm.}

\begin{enumerate}
%where we will chose here $\sigma  = 10$, $\beta = 8/3 $ and $\rho$, proportional to $\Delta T$ will be the parameter to vary (aka test different values).
\item Work on your Homework 7 as a group. Take advantage of this common time to make the most of it (recall it is due by April 14th at 11:59pm). Take this time to get familiar with the problem and assign roles.
\item Solve a $5 \times 5$ linear system of the form $A x = b$, with $A$, $b$ of your choice ($A$ can't be the identity matrix, and should have most of its entry non zero). Compare with the solution obtained while computing directly $x =A^{-1}b $ with the one obtained with the \texttt{conjugate\_gradient} function 
\begin{verbatim}
def conjugate_gradient(A, b, x0, eta, Imax):
    x = x0
    r = b - A@x
    d = r 
    n0 = np.dot(r,r)
    n1 = n0
    res = []
    i = 0
    while n0 > eta, and i < Imax:
        z = A@d
        alpha = n0 / (np.dot(z,d))
        x = x + alpha * d
        r = r - alpha * z
        n1 = np.dot(r,r)
        beta = n1 / n0
        d = r + beta * d
        n0 = n1
        i = i+1
        res = [res, np.sqrt(n1)]
        if np.sqrt(n1) > 1e10:
            break
    return x, i, res
\end{verbatim}
Provide comments on your choices of parameters, what you observe overall (computational time, error, etc.).
%\begin{enumerate}
\item Submit your work (\textbf{only item 2.}) on Catcourses under the assignment \texttt{Worksheet 10} \textbf{as a .pdf}. You will submit your work for item 1. within the assignment \texttt{Homework 7 (group)} in two weeks.
\end{enumerate}

\end{document}
