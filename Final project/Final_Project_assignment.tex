\documentclass[11pt]{article}
\usepackage[normalem]{ulem}
\usepackage{color,hyperref}
\hypersetup{colorlinks,breaklinks,
            linkcolor=blue,urlcolor=blue,
            anchorcolor=lue,citecolor=blue}

\usepackage{amsmath, verbatim}
\usepackage{amssymb}
\usepackage{fancyhdr}
\usepackage{pstricks}
\usepackage{pstricks-add}


\setlength{\oddsidemargin}{0.0in}
\setlength{\evensidemargin}{0.0in}
\setlength{\textheight}{8.4in}
\setlength{\textwidth}{6.5in}
\setlength{\voffset}{0.00in}
\setlength{\headsep}{26pt}
\setlength{\parindent}{0pt}
\setlength{\parskip}{6pt}

% header information
\pagestyle{fancyplain}
\lhead{MATH 150: Wednesday January 27}
\rhead{\textit{Spring Semester 2021}}



\chead{\textbf{Final Project + Videos}\ \ \ \ Due May 7th}

\begin{document}
Your final project will consist of a written portion (Final project report), and two videos (mid-semester video, final project video presentation). 
\section{Final project timeline}

\begin{itemize}
\item Week 7 (March 1-5): teams will be formed and start discussing projects. \textbf{A group leader will be designated and will be responsible to send weekly emails to Prof. Carvalho}.
\item Week 8 (March 8-12): update from group leaders to Prof. Carvalho.
\item Week 9 (March 15-19): \textbf{mid-semester video is due by Friday March 19h at 11:59pm}. Details are given below. This should help set a starting project for each group.
\item Weeks 11-12 : weekly update from group leaders to Prof. Carvalho. If needed make appointments to narrow down a subject.
\item Weeks 14-16: Discussion sections 13-15 are dedicated to work on the project and the report. The TA will assist you if you have questions.
\item Week 16 (May 3-7): \textbf{Reports are due Friday May 7th at 11:59pm}.
\item Week 17 (May 12-14): \textbf{Video project presentation are due May 13th at 12pm}.
\end{itemize}

Every team is highly encouraged to use office hours as your project develops. As a reminder, this project (report + mid-semester video + video project presentation) replaces the final exam for Math 150, and is worth 35\% of your final grade. 

\section{Final project report (15 \% of final grade)}

Your written report should contain the following sections: \\
0) A title page: title, class, names, faculty mentor, and a picture.\\
1) Introduction/motivation: Identify a real system to model mathematically. Present the important features of this system. \\
2) Model: Develop a model that captures those features, at least approximately. Your model should use at least one of the mathematical techniques that you learned in Math 150 this semester. \\
4) Analytic results: Solve your model analytically (if applicable). \\
5) Numerical results and discussion: Solve your model  numerically, discuss the relevance of your results and present the strengths and weaknesses of your model. \\
6) Conclusion: make a summary of your work and mention some possible improvements and extensions for your model.\\
7) Reference section.

Sections 0), 1), 2), 3), 4), 5), 7) can be written as a group, but \textbf{section 6) should be written individually}. Each student will provide an individual report (typed pdf or Jupyter notebook) on Catcourses by Friday May 7th at 11:59pm. \\
%Your oral presentation will take place in class and should last approximately 15 minutes (5 minutes per member). Each member of the team must present a portion of your project (repartition is up to you). You must also prepare slides, that you will turn in together with your written reports. {\bf Attendance to all presentations is mandatory.} \\


\section{Mid-semester Video (10\% of final grade)}

The purpose of the Mid-semester video is to create a presentation of what your group final project subject would be. 
\begin{itemize}
\item If you have found a common problem of interest as a group, each group member can present the same problem but \textbf{the video must be done individually}.
\item If you have not found a common problem as a group, each member can present a different project that would help you narrow down, as a group, your choice.
\end{itemize}
For ease in submission, we are asking students to upload their video to YouTube as an unlisted video and to submit a link to this video. (Details are given below.)\\
 \textbf{What is Required:}
\begin{itemize}
\item 2 - 3 minute video presenting the model problem.
\item We do not expect you to spend more than 2 hours preparing and producing this video. 
\item Your target audience is another undergraduate student in the course. Do not assume knowledge that would be known outside the material of the course.
\item Your video should have the following five components: 
\begin{enumerate}
\item Your video must begin by showing a slide (or other visual aid) and saying out loud each of the following: with your name, your group for the final project, our course title (Math 150), the semester (Spring 2021) and your topic.
\item You video must introduce the real-world problem you want to model in plain language ( + visual aids). This represents the motivation to investigate this modeling problem.
\item You video must introduce the model you think might be relevant (equations if applicable). 
\item You video must list the mathematical tools (analytic and numerical) you may need to use to solve it (do not solve it).
%\item You must go through at least one example of calculation that illustrates the concept.
%\item You must talk about how the idea/topic relates to the rest of the world. 
\item We recommend that you include a slide at the end with your references (see below for more details on references).
\end{enumerate}
\item Your presentation must include visual aids. The easiest way to achieve this is to make slides and record yourself giving a slide show presentation on Zoom. We recommend 2-3 additional slides on top of your introductory slide and references slide. 
\item You can use \textbf{any references you like!} Your course textbook, other textbooks/online resources, other students in the course, Wikipedia, etc. But you must \textbf{explicitly state/list all references you used in making your video!} Please note that conversations with another person are a reference and should be cited. We recommend that you include a slide at the end with your references.
\item It is highly desirable that you show yourself, but if you feel very strongly you do not have to.
\end{itemize}

\section{Final project video presentation (10\% of final grade)}

The purpose of the final project video presentation is to provide a presentation of your final project.
 \textbf{What is Required:}
\begin{itemize}
\item A 15 minute video presenting your final project. Each team member is required to speak the same amount. For example, for a group of three, each student is expected to present 5 minutes (3 x 5 = 15).
\item We recommend to record the video via Zoom: all group members log in to the same meeting, and present back to back (to avoid video editing).
\item We do not expect you to spend more than 2-3 hours preparing and producing this video. 
\item Your target audience is undergraduate students of the class. Do not assume knowledge that would be known outside the material of the course.
\item Your video should follow the same structure of the Final Project Report (sections 0)-7)). Do not present all results you may have found, select what would be relevant (and fitting 15 minutes).
\item Your presentation must include visual aids. You are allowed to use your report as visual aid, the jupyter notebook you created to show real-time simulations, and/or you can make slides.
\item It is highly desirable that you show yourself, but if you feel very strongly you do not have to.
\end{itemize}

\section{Technical advice for video making:}

You are not required to use Zoom to record your video! But it is a tool that you all have access to and it's easy to use.
You simply use Zoom to start a meeting, share your screen (so we can see your slides) and then hit record and start talking. You may have to ensure the Local Recording option is enabled on Zoom: 
\begin{center}
\textbf{How to Make a Recording in Zoom}\\
\url{https://support.zoom.us/hc/en-us/articles/201362473-Local-recording}\\
\url{https://youtu.be/lZHSAMd89JE}
\end{center}

For ease in submission, we are asking that your video projects be uploaded to YouTube as an unlisted video (i.e., it's viewable only by people who you give the link to and is not searchable). You will then submit a link to this video through CatCourses.
YouTube is easy to use if you have a Gmail account (and a Gmail account is free and easy to get!)
Uploading a video to YouTube is \textbf{easy but takes time}. We highly encourage you to leave plenty of time for uploading your video!
\begin{center}
\textbf{How to Upload an Unlisted Video to YouTube:}\\
\url{https://www.youtube.com/watch?v=jaftEW9WI3U}
\end{center}
%\begin{itemize}
%\item \textbf{Most Desirable:} Upload your video to YouTube. You can make the video unlisted (i.e., it's not searchable.) You will submit a link to your video to CatCourses.
%\item You can upload your video to your UC Merced Box Account. Make your video ``sharable'' and submit a link to your file on Box to CatCourses.
%\item \textbf{Least Desirable:} You can upload the video to CatCourses (there is a 2Gb file limit.)
%\end{itemize}

%\section*{Video Rubric: (20 Points)}
%\begin{itemize}
%\item (3 Points): Speaker gives the topic and introduces (name, course, section, semester and topic).
%\item (5 Points): Concept is introduced clearly.
%\item (5 Points): Details of the concept or an example are given clearly. 
%%\item (5 Points): Example problem is solved or worked through clearly.
%%\item (5 Points): Concept chosen is connected back to the world beyond the course. (Why is this useful.)
%\item (3 Points): Effort put into presentation.
%\item (2 Points): Provides visually or verbally (explicitly states) any references used in preparing their project. We recommend including a slide at the end for this. 
%\item (2 Points): Video is between 2-3 minutes.  (0.5 Point deducted for every minute you outside of this range.)
%\end{itemize}
%


\end{document}
